\section*{Introduction}

\begin{frame}{Introduction}
    Most Artificial Intelligence fields involve some sort of \textit{reasoning}: given a \textit{knowledge} an agent reasons and acts rationally towards a \textit{goal}.

    Reasoning for a intelligent artificial agent can be \textit{static} or \textit{dynamic}:
    \begin{itemize}
        \item Static: solve a problem by exploiting some fixed knowledge;
        \item dynamic: solve a problem with mutable knowledge (\eg{} mantaining a property).
    \end{itemize}


    Many applications implicitly involve dynamic reasoning: 
    while such reasoning can be hard, it often is much faster \wrt{} static reasoning.
\end{frame}

\begin{frame}{Introduction (2): static reasoning example}

    \begin{block}{\vphantom{}}
        Given a directed graph $\graph{o} = \paircs{\graphv{}_{0}}{\graphe{o}_{0}}$, check if $\graph{o}$ has at least a cycle
    \end{block}

    \begin{figure}
        \begin{subfigure}{0.45\textwidth}
            \begin{tikzpicture}
                \tikzstyle{vertex} = [shape=circle, fill=blue!20, draw=blue!50, thick, minimum size=2mm, inner sep=5pt, distance=1.7cm];

                \node[vertex](X) at (23:1.5cm) {$x$};
                \node[vertex](Y) at (113:1.5cm) {$y$};
                \node[vertex](Z) at (203:1.5cm) {$z$};
                \node[vertex](T) at (293:1.5cm) {$t$};

                \draw[-{Latex[length=2mm,width=3mm]}, line width=0.3mm]
                    (X) edge node[below]{} (Y)
                    (Y) edge node[left]{} (Z)
                    (X) edge node[below]{} (Z)

                    (Z) edge node[below]{} (T)
                ;
            \end{tikzpicture}
            \caption{$\graph{o}$ does not have a cycle}
        \end{subfigure}\hfill%
        \begin{subfigure}{0.45\textwidth}
            \begin{tikzpicture}
                \tikzstyle{vertex} = [shape=circle, fill=blue!20, draw=blue!50, thick, minimum size=2mm, inner sep=5pt, distance=1.7cm];

                \node[vertex](X) at (23:1.5cm) {$x$};
                \node[vertex](Y) at (113:1.5cm) {$y$};
                \node[vertex](Z) at (203:1.5cm) {$z$};
                \node[vertex](T) at (293:1.5cm) {$t$};

                \draw[-{Latex[length=2mm,width=3mm]}, line width=0.3mm]
                    (X) edge[color=red] node[below]{} (Y)
                    (Y) edge[color=red] node[left]{} (Z)
                    (Z) edge[color=red] node[below]{} (X)

                    (Z) edge node[below]{} (T)
                ;
            \end{tikzpicture}
            \caption{$\graph{o}$ has a cycle}
        \end{subfigure}
    \end{figure}

\end{frame}

\begin{frame}{Introduction (3): dynamic reasoning example}

    \begin{block}{\vphantom{}}
        Given a directed acyclic graph $\graph{o}_{0} = \paircs{\graphv{}_{0}}{\graphe{o}_{0}}$, check if $\graph{o}_{i+1} = \paircs{\graphv{}_{0}}{\graphe{o}_{i} \cup \{(u_i,v_i)\}} \; (u_i,v_i \in \graphv{}_{0}, (u_i,v_i) \not \in \graphe{o}_{i}, i = 0, 1, ..., k)$ has at least a cycle
    \end{block}

    \begin{figure}
        \centering
        \begin{subfigure}{0.29\textwidth}
            \centering
            \begin{tikzpicture}
                \tikzstyle{vertex} = [shape=circle, fill=blue!20, draw=blue!50, thick, minimum size=2mm, inner sep=5pt, distance=1.7cm];

                \node[vertex](X) at (23:0.9cm) {$x$};
                \node[vertex](Y) at (113:0.9cm) {$y$};
                \node[vertex](Z) at (203:0.9cm) {$z$};
                \node[vertex](T) at (293:0.9cm) {$t$};

                \draw[-{Latex[length=2mm,width=3mm]}, line width=0.3mm]
                    (X) edge node[below]{} (Y)
                    (Y) edge node[left]{} (Z)
                    (Z) edge node[below]{} (T)
                ;
            \end{tikzpicture}
            \caption{$\graph{o}_{0}$ does not have a cycle}
        \end{subfigure}%
        \begin{subfigure}{0.04\textwidth}
            \centering
            \scalebox{1.5}{$\Rightarrow$}%
        \end{subfigure}%
        \begin{subfigure}{0.29\textwidth}
            \centering
            \begin{tikzpicture}
                \tikzstyle{vertex} = [shape=circle, fill=blue!20, draw=blue!50, thick, minimum size=2mm, inner sep=5pt, distance=1.7cm];

                \node[vertex](X) at (23:0.9cm) {$x$};
                \node[vertex](Y) at (113:0.9cm) {$y$};
                \node[vertex](Z) at (203:0.9cm) {$z$};
                \node[vertex](T) at (293:0.9cm) {$t$};

                \draw[-{Latex[length=2mm,width=3mm]}, line width=0.3mm]
                    (X) edge node[below]{} (Y)
                    (Y) edge node[left]{} (Z)
                    (Z) edge node[below]{} (T)
                    (X) edge node[below]{} (T)
                ;
            \end{tikzpicture}
            \caption{$\graph{o}_{1}$ does not have a cycle ($u_0 = x, v_0=t$)}
        \end{subfigure}%
        \begin{subfigure}{0.04\textwidth}
            \centering
            \scalebox{1.5}{$\Rightarrow$}%
        \end{subfigure}%
        \begin{subfigure}{0.29\textwidth}
            \centering
            \begin{tikzpicture}
                \tikzstyle{vertex} = [shape=circle, fill=blue!20, draw=blue!50, thick, minimum size=2mm, inner sep=5pt, distance=1.7cm];

                \node[vertex](X) at (23:0.9cm) {$x$};
                \node[vertex](Y) at (113:0.9cm) {$y$};
                \node[vertex](Z) at (203:0.9cm) {$z$};
                \node[vertex](T) at (293:0.9cm) {$t$};

                \draw[-{Latex[length=2mm,width=3mm]}, line width=0.3mm]
                    (X) edge[color=red] node[below]{} (Y)
                    (Y) edge[color=red] node[left]{} (Z)
                    (Z) edge[color=red] node[below]{} (X)

                    (Z) edge node[below]{} (T)
                    (X) edge node[below]{} (T)
                ;
            \end{tikzpicture}
            \caption{$\graph{o}_{2}$ has a cycle ($u_1 = z, v_1=x$).}
        \end{subfigure}
    \end{figure}

    the dynamic problem can be solved with $k$ runs of the algorithm solving the static problem
    
    \begin{center}
    \textbf{however}
    \end{center}
    
    some knowledge is \textbf{not} exploited (\ie{} graphs are cumulative).
\end{frame}

\begin{frame}{Dynamic Graph-based Reasoning Systems}

    The thesis investigates two specific topics involving knowledge represented through graphs:

    \begin{itemize}
        \item \textbf{Consistency checking problem in temporal reasoning}: check if a network of constraint encoding temporal information is satisfiable;
        \item \textbf{\pathfinding{}}: given a graph, find the shortest path from a start vertex to a target.
    \end{itemize}

    For both of them, we have considered a dynamic variant of the problem and propose efficient algorithms for solving such variant. 
    The algorithms have been experimentally evaluated showing significant gains \wrt{} \stateofart{}.
\end{frame}

\begin{frame}{Talk Outline}

    The talk will be outlined as follows
    \begin{itemize}
        \item Decremental Consistency Checking Problem:
        \begin{itemize}
            \item Background (\CSPName{}, consistency, \tlGraphName{}, temporal algebras);
            \item motivation and problem definition;
            \item \DPASATAlgorithmName{} and \DOHSATAlgorithmName{};
            \item experimental results;
        \end{itemize}
        \item \SAPFEC{} Problem:
        \begin{itemize}
            \item Background (\pathfinding{}, \CPD{});
            \item motivation and problem definition;
            \item \CPDSearch{};
            \item experimental results;
        \end{itemize}
        \item conclusion and future works;
    \end{itemize}
\end{frame}